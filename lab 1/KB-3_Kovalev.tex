\documentclass[12pt,a4paper]{article}
\usepackage{amsmath,amssymb,amsthm}
\usepackage{algorithm}
\usepackage[noend]{algpseudocode} 
\usepackage[utf8]{inputenc}
\usepackage[russian]{babel}
\usepackage{geometry}
\usepackage{listings}
\usepackage{xcolor}
\lstset{
    basicstyle=\ttfamily\small,
    keywordstyle=\color{blue},
    commentstyle=\color{green},
    numbers=left,
    frame=single,
    breaklines=true
}
\geometry{
    a4paper,
    left=30mm,
    right=15mm,
    top=20mm,
    bottom=20mm
}

% PROBABILITY SYMBOLS
\newcommand*\PROB\Pr 
\DeclareMathOperator*{\EXPECT}{\mathbb{E}}

% Sets, Rngs, ets 
\newcommand{\N}{{{\mathbb N}}}
\newcommand{\Z}{{{\mathbb Z}}}
\newcommand{\R}{{{\mathbb R}}}
\newcommand{\Zp}{\ints_p} % Integers modulo p
\newcommand{\Zq}{\ints_q} % Integers modulo q
\newcommand{\Zn}{\ints_N} % Integers modulo N

% Landau 
\newcommand{\bigO}{\mathcal{O}}
\newcommand*{\OLandau}{\bigO}
\newcommand*{\WLandau}{\Omega}
\newcommand*{\xOLandau}{\widetilde{\OLandau}}
\newcommand*{\xWLandau}{\widetilde{\WLandau}}
\newcommand*{\TLandau}{\Theta}
\newcommand*{\xTLandau}{\widetilde{\TLandau}}
\newcommand{\smallo}{o} %technically, an omicron
\newcommand{\softO}{\widetilde{\bigO}}
\newcommand{\wLandau}{\omega}
\newcommand{\negl}{\mathrm{negl}} 

% Misc
\newcommand{\eps}{\varepsilon}
\newcommand{\inprod}[1]{\left\langle #1 \right\rangle}

\newcommand{\handout}[5]{
  \noindent
  \begin{center}
  \framebox{
    \vbox{
      \hbox to 5.78in { {\bf Учебная практика (учебно-лабораторная)} \hfill #2 }
      \vspace{4mm}
      \hbox to 5.78in { {\Large \hfill #5  \hfill} }
      \vspace{2mm}
      \hbox to 5.78in { {\em #3 \hfill #4} }
    }
  }
  \end{center}
  \vspace*{4mm}
}

\newcommand{\lecture}[4]{\handout{#1}{#2}{#3}{Автор: #4}{Название темы #1}}

\newtheorem{theorem}{Теорема}
\newtheorem{lemma}{Лемма}
\newtheorem{definition}{Определение}
\newtheorem{corollary}{Следствие}
\newtheorem{fact}{Факт}

\begin{document}

\begin{titlepage}
    \centering
    {\large\textbf{Федеральное государственное автономное образовательное учреждение высшего образования}}\par
    \vspace{0.5cm}
    {\large\textbf{Балтийский федеральный университет имени Иммануила Канта}}\par
    \vspace{0.5cm}
    
    {\large Институт физико-математических наук и информационных технологий}\par
    \vspace{3cm}
    
    {\Large\textbf{ОТЧЕТ}}\par
    \vspace{1cm}
    
    {\Large по учебной практике (учебно-лабораторной)}\par
    \vspace{2cm}
    
    \begin{flushright}
        \large
        \textbf{Выполнил:}\par
        студент \underline{Ковалев М.В.}\par
        \vspace{0.5cm}
        группы \underline{05-КБ-22-О}\par
        \vspace{0.5cm}
        направление подготовки 10.05.01 <<Компьютерная безопасность>>\par
        \vspace{2cm}
        
        \textbf{Руководитель практики:}\par
        ст. преподаватель \underline{Глинчиков К.Е.}\par
        \vspace{2cm}
    \end{flushright}
    
    \begin{flushright}
        Работа выполнена:  <<\underline{\hspace{1cm}}>> \underline{\hspace{3cm}} 2025 г.\par
        \vspace{1cm}
        Работа защищена с оценкой: \underline{\hspace{6cm}}\par
        \vspace{1cm}
        <<\underline{\hspace{1cm}}>> \underline{\hspace{3cm}} 2025 г.\par
    \end{flushright}
    
    \vspace{2cm}
    {\large Калининград 2025}
\end{titlepage}

\newpage

\lecture{Вычисление символа Якоби}{6 семестр 2025 г.}{}{Ковалев М.В.}

\section{Введение}

Символ Якоби — важная теоретико-числовая функция, используемая в криптографии и теории чисел для определения квадратичных вычетов по модулю нечётного числа без необходимости его факторизации. В данной работе рассматривается алгоритм вычисления символа Якоби и его применение в компьютерной безопасности.

\section{Основные определения}

\begin{definition}[Символ Лежандра]
Для простого нечётного числа $p$ и целого числа $a$ символ Лежандра $\left(\frac{a}{p}\right)$ определяется как:
\[
\left(\frac{a}{p}\right) = 
\begin{cases}
0, & \text{если } p \mid a \\
1, & \text{если } a \text{ — квадратичный вычет по модулю } p \\
-1, & \text{иначе}
\end{cases}
\]
\end{definition}

\begin{definition}[Символ Якоби]
Для нечётного натурального $n = p_1^{\alpha_1}\cdots p_k^{\alpha_k}$ и целого $a$ символ Якоби определяется как:
\[
\left(\frac{a}{n}\right) = \prod_{i=1}^k \left(\frac{a}{p_i}\right)^{\alpha_i}
\]
\end{definition}
\section{Свойства символа Якоби}

Символ Якоби обладает следующими фундаментальными свойствами:

\begin{theorem}[Основные свойства символа Якоби]
Для любых целых чисел $a, b$ и нечётных натуральных чисел $m, n$ справедливо:
\begin{enumerate}
\item \textbf{Мультипликативность}:
\[ \left(\frac{ab}{n}\right) = \left(\frac{a}{n}\right)\left(\frac{b}{n}\right) \]
\[ \left(\frac{a}{mn}\right) = \left(\frac{a}{m}\right)\left(\frac{a}{n}\right) \]

\item \textbf{Периодичность по верхнему аргументу}:
Если $a \equiv b \pmod{n}$, то $\left(\frac{a}{n}\right) = \left(\frac{b}{n}\right)$

\item \textbf{Вычисление для -1}:
\[ \left(\frac{-1}{n}\right) = (-1)^{\frac{n-1}{2}} = 
\begin{cases}
1, & \text{если } n \equiv 1 \pmod{4} \\
-1, & \text{если } n \equiv 3 \pmod{4}
\end{cases} \]

\item \textbf{Вычисление для 2}:
\[ \left(\frac{2}{n}\right) = (-1)^{\frac{n^2-1}{8}} = 
\begin{cases}
1, & \text{если } n \equiv \pm 1 \pmod{8} \\
-1, & \text{если } n \equiv \pm 3 \pmod{8}
\end{cases} \]

\item \textbf{Квадратичный закон взаимности} (для взаимно простых $m, n \geq 3$):
\[ \left(\frac{m}{n}\right) = (-1)^{\frac{(m-1)(n-1)}{4}} \left(\frac{n}{m}\right) \]
\end{enumerate}
\end{theorem}

\begin{proof}[Доказательство свойств]
Докажем некоторые из этих свойств:

1. \textbf{Мультипликативность} следует непосредственно из определения символа Якоби как произведения символов Лежандра.

3. \textbf{Для -1}: Рассмотрим простой случай, когда $n$ — простое число. По критерию Эйлера:
\[ \left(\frac{-1}{p}\right) \equiv (-1)^{\frac{p-1}{2}} \pmod{p} \]
Для составного $n$ результат следует из мультипликативности.

4. \textbf{Для 2}: Аналогично, для простого $p$:
\[ \left(\frac{2}{p}\right) \equiv (-1)^{\frac{p^2-1}{8}} \pmod{p} \]
что можно проверить непосредственным вычислением для различных классов вычетов.
\end{proof}
\section{Алгоритм вычисления}

\begin{algorithm}[h]
\caption{Вычисление символа Якоби}
\label{alg:Jacobi}
\textbf{Вход:} Нечётное $n > 0$, целое $a$ \\
\textbf{Выход:} $\left(\frac{a}{n}\right)$

\begin{algorithmic}[1]
\State $result \gets 1$
\While{$a \neq 0$}
    \While{$a \equiv 0 \pmod{2}$}
        \State $a \gets a/2$
        \State $r \gets n \mod 8$
        \If{$r = 3$ или $r = 5$} $result \gets -result$ \EndIf
    \EndWhile
    \State Поменять местами $a$ и $n$
    \If{$a \equiv n \equiv 3 \pmod{4}$} $result \gets -result$ \EndIf
    \State $a \gets a \mod n$
\EndWhile
\If{$n = 1$} \Return $result$ \Else \Return $0$ \EndIf
\end{algorithmic}
\end{algorithm}

\section{Теоретический анализ}

\begin{theorem}[Квадратичный закон взаимности]
Для взаимно простых нечётных $m,n \geq 3$:
\[
\left(\frac{m}{n}\right)\left(\frac{n}{m}\right) = (-1)^{\frac{(m-1)(n-1)}{4}}
\]
\end{theorem}

\begin{theorem}
Алгоритм \ref{alg:Jacobi} корректно вычисляет символ Якоби за $O(\log n)$ арифметических операций.
\end{theorem}

\begin{proof}
Корректность следует из свойств символа Якоби и квадратичного закона взаимности. Сложность определяется аналогично алгоритму Евклида.
\end{proof}

\section{Пример вычисления}

Вычислим $\left(\frac{1001}{9907}\right)$:

\begin{align*}
\left(\frac{1001}{9907}\right) &= -\left(\frac{9907}{1001}\right) = -\left(\frac{900}{1001}\right) \\
&= -\left(\frac{4}{1001}\right)\left(\frac{225}{1001}\right) = -\left(\frac{225}{1001}\right) \\
&= -\left(\frac{1001}{225}\right) = -\left(\frac{101}{225}\right) \\
&= -\left(\frac{225}{101}\right) = -\left(\frac{23}{101}\right) \\
&= \left(\frac{101}{23}\right) = \left(\frac{9}{23}\right) = 1
\end{align*}
\section{Сравнение скорости работы алгоритмов}

Для сравнения эффективности нашей реализации алгоритма вычисления символа Якоби со встроенной функцией в SageMath был проведён ряд тестов на больших входных данных. Результаты представлены в таблице \ref{table:speed_comparison}.

\begin{table}[h]
\centering
\caption{Сравнение времени работы алгоритмов (в секундах)}
\label{table:speed_comparison}
\begin{tabular}{|c|c|c|c|}
\hline
Битовая длина $n$ & Наша реализация & SageMath & Ср.Отношение \\ 
\hline
256 бит & 0.00097 & 0.0001 & 6 \\
512 бит  & 0.00081 & 0.000068 & 11 \\
1024 бит  & 0.0015 &  0.000076 & 21 \\
2048 бит  & 0.003 & 0.0001 & 32 \\

\hline
\end{tabular}
\end{table}

\subsection{Методика тестирования}

Тестирование проводилось на следующей конфигурации:
\begin{itemize}
\item Процессор: Intel Core i5-1235U (1.30 GHz)
\item Память: 16 ГБ DDR4
\item ОС: Windows 11
\item SageMathCell - браузерная ячейка
\end{itemize}

\subsection{Анализ результатов}

Как видно из таблицы \ref{table:speed_comparison}:
\begin{itemize}
\item Встроенная функция SageMath работает сильно быстрее на больших числах
\item Разница в скорости растёт при увеличении размера входных данных

\end{itemize}



\begin{lstlisting}[language=Python,caption=Код для тестирования]
import time
def jacobi(a, n):
    if n <= 0 or n % 2 == 0:
        raise ValueError("n неподходящее")
    a = a % n
    result = 1
    while a != 0:    
        while a % 2 == 0:
            a = a // 2
            r = n % 8
            if r == 3 or r == 5:
                result = -result    
        a, n = n, a
        
        if a % 4 == 3 and n % 4 == 3:
            result = -result
        a = a % n
    if n == 1:
        return result
    else:
        return 0
def test_jacobi_performance(bit_length, trials=1):
    n = random_prime(2^bit_length, lbound=2^(bit_length-1))
    a = randint(2, n-1)
    print(a,n)
    # Тестирование собственной реализации
    start = time.time()
    for _ in range(trials):
        j1 = jacobi(a, n)
    our_time = (time.time() - start)/trials

    # Тестирование SageMath
    start = time.time()
    for _ in range(trials):
        j2 = jacobi_symbol(a, n)
    sage_time = (time.time() - start)/trials
    
    assert j1 == j2  
    print(our_time/sage_time)
    return (our_time, sage_time*1.0)
\end{lstlisting}

\section{Применение в криптографии}

Символ Якоби применяется в:
\begin{itemize}
\item Тестах простоты (Соловея-Штрассена)
\item Криптосистемах с открытым ключом
\item Алгоритмах факторизации
\item Протоколах аутентификации
\end{itemize}

\newpage
\begin{center}
\paragraph{Список литературы}
\end{center}

\begin{enumerate}
\item Коблиц Н. Курс теории чисел и криптографии. — М.: Наука, 2001.
\item Василенко О.Н. Теоретико-числовые алгоритмы в криптографии. — М.: МЦНМО, 2003.
\end{enumerate}

\end{document}
